\apendice{Plan de proyecto software}
\newcommand{\code}[1]{\texttt{#1}}

\section{Introducción}
En este apartado se tiene como objetivo definir y estructurar el plan de proyecto para el desarrollo de un lenguaje de programación y el compilador del mismo, de forma que queden satisfechas sus necesidades y requerimientos técnicos, garantizando la calidad, eficiencia y cumplimiento de unos plazos establecidos. Este plan proporciona un marco de referencia para la gestión del proyecto, incluyendo la planificación temporal así como un estudio de la viabilidad tanto económica como legal.

\section{Planificación temporal}
En este apartado se explica de que forma se aplicará la metodología ágil por excelencia, Scrum, para el control temporal de este proyecto, cabe destacar que al ser un trabajo unipersonal con fin académico puede no reflejar a la perfección el trabajo habitual de esta metodología. 

A continuación se muestra una planificación organizada por \eng{sprints}, donde se prioriza la mejora de un producto mínimo viable (MVP por sus siglas en inglés), de forma que en los primeros \eng{sprints} se obtiene este MVP y en los próximos \eng{sprints} se trabaja en ampliar las funcionalidades incrementalmente. Cada dos \eng{sprints} se realiza una reunión con el tutor, para evaluar el trabajo realizado y planificar los siguientes \eng{sprints}, teniendo en cuenta lo aprendido anteriormente y las posibles dificultades encontradas.

Así mismo, para mantener un orden durante el desarrollo, se integrará en Github la herramienta ZenHub, la cual ayudará a mantener la planificación en un tablero estilo Kanban, el cual se integra con las issues y milestones definidas.

A continuación se citan los \eng{sprints} realizados.

\subsection{1º Milestone: MVP}
\textbf{\eng{Sprint} 0 (15/09/2025 - 04/10/2025):} Se investiga sobre herramientas y conceptos relacionados con los lenguajes y compiladores.

\textbf{\eng{Sprint} 1 (05/10/2025 - 11/10/2025):} Se inicia el repositorio y se trabaja en las primeras dos fases (análisis léxico y sintáctico) del compilador.

\textbf{\eng{Sprint} 2 (12/10/2025 - 18/10/2025):} Primera reunión con el tutor. Se trabaja en la fase de generación de código intermedio del compilador. 

\textbf{\eng{Sprint} 3 (19/10/2025 - 24/10/2025):} Se añade una funcionalidad para visualización del AST, además, se incluyen mejoras y se madura el trabajo realizado en los \eng{sprints} anteriores.   

\subsection{2º Milestone: Lenguaje completo}

\textbf{\eng{Sprint} 4 (25/09/2025 - 01/11/2025):} Primera ampliación del MVP, se añaden tipos, declaraciones y asignaciones de variables.

\textbf{\eng{Sprint} 5 (02/11/2025 - 08/11/2025):} Segunda ampliación del lenguaje, se añaden las funciones y estructura de control \texttt{if-the-else}.

\textbf{\eng{Sprint} 6 (09/11/2025 - 14/11/2025):} Tercera ampliación del lenguaje, se añaden bucles \texttt{while} y \texttt{for}. Se añade la fase del compilador responsable de llevar el LLVM IR a código máquina específico de la arquitectura anfitrión.

\textbf{\eng{Sprint} 7 (15/11/2025 - 21/11/2025):} Se añade un \eng{runtime} mínimo personalizado, se integran funciones de la biblioteca estándar de C como \code{printf}, \code{strlen} e implementa una librería estándar propia con una función \texttt{toString}. Se añade el tipo que modelará el tiempo.

\subsection{3º Milestone: Eventos y tiempo}

\textbf{\eng{Sprint} 8 (22/11/2025 - 28/11/2025):} Se añade la estructura \texttt{event}, que modela la principal característica propia del lenguaje. Además se trabaja en refactorizaciones de calidad y seguridad de la fase de análisis semántico del compilador.

\textbf{\eng{Sprint} 9 (29/11/2025 - 05/12/2025):} Se realizan importantes mejoras de calidad del software. Se añade \eng{shadowing} de variables y se comienza a hacer pruebas sobre la integración del runtime.

\textbf{\eng{Sprint} 10 (06/12/2025 - 12/12/2025):} Se completa una primera versión funcional del \eng{runtime}, se añaden operadores unitarios \texttt{++} y \texttt{--}. Se añade una función print propia del lenguaje.

\textbf{\eng{Sprint} 11 (13/12/2025 - 19/12/2025):} Se mejora la gestión de argumentos con la librería  y de errores del compilador mediante registro de información adicional y el uso de una estructura de datos específica para los errores.

\textbf{\eng{Sprint} 12 (20/12/2025 - 26/12/2025):} Se completa el paso de parámetros mediante \texttt{libffi} entre \eng{runtime} y código compilado. Se añade la fase de compilación. 

\textbf{\eng{Sprint} 13 (27/12/2025 - 03/01/2025):} Refactorizaciones para mejorar la calidad del código. Redacción exhaustiva y revisiones de la memoria y anexos. 

\textbf{\eng{Sprint} 14 (04/01/2025 - 15/01/2025):} El \eng{sprints} final.

\section{Estudio de viabilidad}
En este apartado analizaremos la viabilidad desde el punto de vista de recursos económicos y requisitos para llevarlo a cabo dentro del marco legal. Este análisis puede parecer poco necesario en un proyecto académico, sin embargo, sería vital en un proyecto privado real.

\subsection{Viabilidad económica}
A continuación analizaremos que costes deben ser tenidos en cuenta para el desarrollo de este proyecto, así como una determinación aproximada de los costes específicos y totales.

\textbf{Coste humano:} es el coste del salario que se requiere para el equipo que programara durante el proyecto. Puesto que es un proyecto con un único desarrollador, se estima un sueldo mensual de unos 1500€ brutos.

\textbf{Coste hardware:} este coste representa todo el material que requiere el equipo para funcionar correctamente, este comprende materiales como ordenadores, teclados, ratones, etc. Dando por hecho que en los 5 meses de proyecto solo se alquila el material necesario, podría estimarse un alquiler mensual de 200€ por hardware.

\textbf{Coste software:} estos son costos asociados comúnmente a desarrollos software, sin embargo, con la elección de software libre vamos a evitar este gasto tan común.

\subsection{Coste total}
En este caso de un proyecto de 5 meses, los costes los podemos aproximar de la siguiente forma:

\tablaSmallSinColores{Costes totales.}{lcr}{Coste}{Tipo de coste & Precio aproximado \\}{
  Coste humano & 7500€ \\
  Coste hardware & 1000€  \\
  Coste software & 0€ \\
  \midrule
  \textbf{Coste total} & \textbf{8500€} \\
}

\subsection{Viabilidad legal}
A continuación se citan todos las herramientas y bibliotecas utilizadas junto con las licencias asociadas a las mismas:

\tablaSmallSinColores{Licencias.}{lcr}{Licencias}{Herramienta & Licencia \\}{
  Ubuntu & GPL \\
  ANTLR & BDS  \\
  LLVM & UIUC \\
  GCC & GPLv3 \\
  CMake & BSD \\
  LaTex & LPPL \\
  Doxygen & GPL \\
  Gtest & BDS \\
  argparse lib & MIT \\
  spdlog lib & MIT \\
}

Como podemos observar, todas las licencias de estas herramientas permiten el uso \textbf{gratuito} de las mismas, sin restricciones adicionales que afecten a este proyecto.