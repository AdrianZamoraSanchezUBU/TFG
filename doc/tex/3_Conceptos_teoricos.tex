\capitulo{3}{Conceptos teóricos}

\subsection{Sobre lenguajes de programación}
\begin{description}
    \item [Lenguaje de programación]: Un lenguaje de programación es todo aquel lenguaje que permite escribir instrucciones las cuales, tras compilarse, pueden ser ejecutadas en un sistema informático.
    \item [Compilación]: Es el proceso por el cual un lenguaje de programación se convierte en un programa ejecutable por un computador.
    \item [Clasificación de lenguajes]: Los lenguajes de programación suelen clasificarse según su nivel de abstracción (alto o bajo nivel), paradigma (imperativo, declarativo, OOP, funcional), propósito del lenguaje (propósito general, lenguaje específico del dominio) y por forma de ejecución (compilados o interpretados).  
    \item [Tipado]: El tipado de un lenguaje se refiere a la forma en la que se gestionan los tipos de los datos, en los lenguajes con tipado estático, los tipos deben ser especificados en las declaraciones y se hacen comprobaciones estrictas durante las asignaciones, por otro lado, los lenguajes de tipado dinámico permiten declaraciones sin especificar el tipo, en estos lenguajes normalmente se hacen transformaciones de datos en tiempo de ejecución. 
    \item [Lenguaje máquina]: El lenguaje máquina es aquel que puede ser ejecutado por un ordenador, este tipo de lenguaje suele ser generado por un compilador, puesto que sería extremadamente ineficiente de escribir por un ser humano.
    \item [Scope]: Un alcance o scope se refiere normalmente a la accesibilidad a elementos como variables o funciones desde diferentes partes del código. Por ejemplo, una variable definida dentro de una función solo será accesible desde el bloque interno de la función pero no será accesible desde el bloque de código donde se ha definido la propia función.
\end{description}

\subsection{Sobre compiladores}
\begin{description}
    \item [Compilador por fases]: Es una técnica de construcción de compiladores que destaca por su alta modularidad, cada fase está bien definida y tiene una única responsabilidad, se caracterizan por ser fácilmente modificables y ampliables. 
    \item [Análisis léxico]: Es una fase del proceso de compilación en la cual se analizan las secuencias de caracteres de un texto y se separan en tokens o lexemas. A la herramienta que realiza este proceso se le llama \emph{scaner}, \emph{lexer} o \emph{tokenizer}
    \item [Análisis sintáctico]: Esta fase del proceso de compilación trata de agrupar los tokens obtenidos durante el análisis léxico, para ello se vale de reglas que generan otras estructuras de datos (como AST) desde la cadena de tokens.
    \item [AST]: El \emph{Abstract Syntax Tree} (árbol de sintaxis abstracta), es una estructura de datos que se utiliza comúnmente en compiladores para representar producciones sintácticas, donde los tokens forman una estructura jerárquica de árbol que representa el programa y ayuda a su posterior interpretación.
    \item [Tabla de símbolos]: Es una estructura de datos utilizada en compiladores para asociar cada símbolo de un programa con su ubicación, alcance y tipo de dato. Dentro de un compilador cumple un papel fundamental entre el front end y el back end del compilador.
    \item [Análisis semántico]: Es una fase en la cual el compilador comprueba la corrección de las producciones válidas formadas en el análisis sintáctico, algunas correcciones podrían ser la verificación de tipos y corrección en asignaciones y expresiones. Normalmente en esta fase se separa la información necesaria para rellenar la tabla de símbolos con los datos de los identificadores presentes en el código.
    \item [Generación de código intermedio]: La creación de una representación intermedia o IR por sus siglas en inglés, hace referencia a un código que queda a medio camino entre el código fuente y el código máquina. Especialmente en el caso del IR de LLVM utilizado en este trabajo, esta fase nos permite tener una mejor base para realizar diferentes optimizaciones y conversiones a diferentes formas de lenguaje máquina, según la arquitectura objetivo.
    \item [Optimización]: Esta fase consisten en realizar cambios o mejoras en la forma del lenguaje, sin alterar el significado original. Estas pueden ser eliminaciones de código no alcanzable, redefiniciones de algunas estructuras para evitar generar más variables de las necesarias, gestión eficiente de la memoria, etc.
    \item [Generación de código final]: Esta es la fase final del proceso de compilación, durante este paso se convierte el IR a código máquina, el cual está totalmente listo para ser ejecutado.
    \item [Runtime]: Es el código de soporte que se ejecuta junto a un programa compilado y se encarga de inicializarlo, gestionar recursos y proporcionar un entorno controlado durante su ejecución.
\end{description}