\capitulo{3}{Conceptos teóricos}

\subsection{Sobre lenguajes de programación}
\begin{description}
    \item [Lenguaje de programación]: Lenguaje que permite escribir instrucciones que, tras ser compiladas, pueden ejecutarse en un sistema informático.
    \item [Compilación]: Proceso mediante el cual un lenguaje de programación se transforma en un programa ejecutable por un computador.
    \item [Clasificación de lenguajes]: Los lenguajes pueden clasificarse según su nivel de abstracción, paradigma (imperativo, declarativo, OOP, funcional), propósito (general o específico del dominio) y forma de ejecución (compilados o interpretados).
    \item [Tipado]: Se refiere a cómo se gestionan los tipos de datos. En lenguajes de tipado estático los tipos se especifican en las declaraciones y se comprueban durante la compilación, mientras que en los de tipado dinámico estas comprobaciones y conversiones se realizan en tiempo de ejecución.
    \item [Lenguaje máquina]: Lenguaje directamente ejecutable por el ordenador, normalmente generado por un compilador debido a su alta complejidad para ser escrito manualmente.
    \item [Scope]: El alcance define la accesibilidad de variables o funciones desde distintas partes del código. Por ejemplo, una variable declarada dentro de una función solo es accesible desde su bloque interno.
\end{description}

\subsection{Sobre compiladores}
\begin{description}
    \item [Compilador por fases]: Técnica de construcción basada en módulos bien definidos, donde cada fase tiene una única responsabilidad, facilitando su modificación y ampliación.
    \item [Análisis léxico]: Fase en la que el texto fuente se descompone en tokens o lexemas. La herramienta encargada de este proceso se denomina \emph{lexer} o \emph{scanner}.
    \item [Análisis sintáctico]: Agrupa los tokens obtenidos según reglas gramaticales, generando estructuras como el AST a partir de la secuencia de tokens.
    \item [AST]: El \emph{Abstract Syntax Tree} es una estructura jerárquica que representa el programa y facilita su posterior análisis y transformación.
    \item [Tabla de símbolos]: Estructura que asocia cada identificador del programa con información como su tipo, alcance y ubicación, siendo clave en la comunicación entre el fases.
    \item [Análisis semántico]: Fase encargada de comprobar la corrección semántica del programa, como la verificación de tipos o asignaciones, y de recopilar la información necesaria para la tabla de símbolos.
    \item [Generación de código intermedio]: Producción de una representación intermedia (IR) situada entre el código fuente y el código máquina. En este proyecto se emplea LLVM IR, lo que facilita optimizaciones y la generación de código para distintas arquitecturas.
    \item [Optimización]: Consiste en mejorar el código generado sin modificar su significado, aplicando técnicas como eliminación de código no alcanzable o uso más eficiente de memoria.
    \item [Generación de código final]: Fase en la que el IR se transforma en código máquina listo para su ejecución.
    \item [Runtime]: Código de soporte que acompaña al programa compilado y se encarga de su inicialización, gestión de recursos y ejecución.
\end{description}