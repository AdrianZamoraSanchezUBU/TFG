\capitulo{1}{Introducción}
En la informática moderna estamos acostumbrados a disponer de numerosos lenguajes de programación, los cuales empleamos sin detenernos a analizar más allá de sus potenciales usos, el rendimiento que nos ofrecen o las librerías de las que dispone. Sin embargo, estas complejas herramientas cuentan con un gran trabajo detrás, el cual permite a los programadores abstraerse de complejos conceptos sobre lenguajes o computadores, pudiendo dedicar su tiempo únicamente a la tarea de la programación.

La creación de los compiladores es una ciencia que ha evolucionado mucho a lo largo de los años, no solo por el surgimiento de nuevos paradigmas de lenguajes y sus compiladores asociados, muchos lenguajes que llevan con nosotros desde los 70s han experimentado cambios significativos en sus sintaxis y compiladores, además de versiones alternativas, cada una enfocada en unos objetivos diferentes que se traduce a nuevos retos y formas de entender los lenguajes y sus procesos de compilación.

Ejemplos de proyectos que comenzaron siendo pequeños lenguajes de programación o lenguajes de dominio concreto que han crecido hasta ser herramientas ampliamente utilizadas son: 
\begin{itemize}
    \item \textbf{HTML + CSS}: HTML fue creado por Tim Berners-Lee a principios de los años 90 como el lenguaje estándar para estructurar documentos en la World Wide Web, permitiendo hipervínculos y texto marcado; posteriormente, CSS fue propuesto por Håkon Wium Lie y estandarizado en 1996 por W3C con el objetivo de separar claramente el contenido de su presentación, enriqueciendo enormemente el diseño visual de las páginas web. Hoy día disponen de importantes \eng{frameworks} y su impacto ha inspirado llegado a otros lenguajes y herramientas fuera del entorno web. \cite{html5, css_standard1996}
    \item \textbf{MATLAB}: Diseñado en los 80 por Jack Little y Cleve Moler para ser un lenguaje específico de álgebra lineal. Hoy día ha crecido como una herramienta ampliamente utilizada en ingeniería, robótica, procesamiento de señales, visión por computadora, etc. \cite{matlab}
    \item \textbf{SQL}: Comenzó en los 70 como un lenguaje que IBM desarrollaría para hacer consultas en sus bases de datos relacionales \cite{wiki:sql_ibm}, sin embargo, fue adoptado como un estándar ISO/IEC \cite{iso2016sql} en la consulta de datos, que lo ha llevado a estar presente en prácticamente todos los motores de bases de datos relaciones modernos.
    \item \textbf{JavaScript}: Lenguaje creado a mediados de los 90, cuando los navegadores y entornos web empiezan a despegar. Su primera versión la hace Brendan Eich \cite{wiki:javascript} en tan solo diez días, su primer nombre es "Mocha" pero por cuestión de marketing deciden aprovechar la fama de Java y lo llaman JavaScript. Con el crecimiento de la web y los navegadores termina siendo estandarizado bajo el estándar ECMAScript \cite{js_ecmascript1997}.
    \item \textbf{Python}: Guido Van Rossum comenzó su desarrollo a finales de los 80, con el único objetivo de crear un lenguaje que "fácil de usar, incluso para programadores no profesionales" \cite{python_history}. En su primera versión es adoptado por numerosas universidades como herramienta científica. El lenguaje ha crecido hasta su versión 3, siendo uno de los lenguajes más usados, con numerosos módulos que lo convierten en una herramienta extremadamente versátil.
\end{itemize}

\subsection{Memoria}
La memoria ofrece una visión completa del proyecto, incluyendo la definición de objetivos, el marco teórico necesario para entender el presente documento y sus anexos, las técnicas y herramientas utilizadas, así como los aspectos más relevantes del desarrollo. También se enmarca este trabajo con otros proyectos o trabajos relacionados que han servido como una base para facilitar el conocimiento acerca de este área de estudio. Finalmente, se presentan las conclusiones obtenidas, así como posibles líneas de trabajo futuras donde se sugieren mejoras o ampliaciones del trabajo realizado.

\subsection{Anexos}
En el apartado de anexos, se complementa la información de la memoria, proporcionando una documentación más detallada del proyecto, como el plan de proyecto, las especificación de requisitos, el diseño del sistema, manuales de programación, manuales de uso para el usuario final y, finalmente, la justificación de su valor educativo y curricular.