\capitulo{1}{Introducción}
En la informática moderna estamos acostumbrados a disponer de numerosos lenguajes de programación, los cuales empleamos sin detenernos a analizar más allá de sus potenciales usos y librerías disponibles. Sin embargo, estas complejas herramientas cuentan con un gran trabajo detrás, el cual permite a los programadores utilizar estas herramientas sin preocuparse de cómo funcionan realmente.

La creación de los compiladores es una ciencia que ha evolucionado mucho a lo largo de los años, no solo por el surgimiento de nuevos lenguajes y sus compiladores asociados, muchos lenguajes que llevan con nosotros desde los 70s han experimentado cambios significativos en sus compiladores, además de versiones alternativas, cada una enfocanda en unos objetivos diferentes.

Ejemplos de proyectos que comenzaron siendo pequeños lenguajes de programación o lenguajes de dominio concreto que han crecido hasta ser herramientas ampliamente utilizadas son: 
\begin{itemize}
    \item \textbf{HTML + CSS}: Nacieron como lenguajes muy específicos que hoy día son estándares globales, tienen un ecosistema inmenso, disponen de importantes frameworks e incluso han inspirado a lenguajes y herramientas fuera del entorno web.
    \item \textbf{MATLAB}: Diseñado en los 80s para ser un lenguaje específico de algebra lineal, que hoy día se utiliza ampliamente en ingeniería, robótica, procesamiento de señales, visión por computadora, etc.  
    \item \textbf{SQL}: Comenzó en los 70s como un lenguaje que IBM desarrollaría para hacer consultas en sus bases de datos relacionales, sin embargo, fue ampliamente adoptado como un estándar que lo ha llevado a estar presente en prácticamente todos los motores de bases de datos relaciones modernos. 
\end{itemize}

\subsection{Memoria}
Esta memoria aborda los siguientes apartados:
\begin{itemize}
    \item \textbf{Objetivos}:
    \item \textbf{Conceptos teóricos}:
    \item \textbf{Técnicas y herramientas}:
    \item \textbf{Aspectos relevantes del desarrollo del proyecto}:
    \item \textbf{Trabajos relacionados}:
    \item \textbf{Conclusiones y líneas de trabajo futuras}:
\end{itemize}

\subsection{Anexos}
Adicionalmente, se adjuntan los siguientes anexos:
\begin{itemize}
    \item \textbf{Anexo A. Plan de proyecto}:
    \item \textbf{Anexo B. Requisitos}:
    \item \textbf{Anexo C. Diseño}:
    \item \textbf{Anexo D. Manual de programador}:
    \item \textbf{Anexo E. Manual de usuario}:
    \item \textbf{Anexo F. Sostenibilización curricular}:
\end{itemize}