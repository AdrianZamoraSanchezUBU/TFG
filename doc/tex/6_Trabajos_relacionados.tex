\capitulo{6}{Trabajos relacionados}
En el aspecto más teórico destaca el Libro del Dragon de Aho, Sethi & Ullman, el cual es un referente en cuanto a la teoría de lenguajes, lexers, parsers, construcciones de AST, analizadores semánticos, generación de IR y optimizaciones. Aunque es un material muy completo, su extensión me ha llevado a consultar apartados concretos y resumenes enfocados en ciertos aspectos relevantes para mi proyecto.

Destaca como proyecto de alcance y complejidad similar el tutorail/lenguaje didactivo llamado Kleidoscope el cual es parte del tutorial oficial de LLVM sobre compiladores. El proyecto Kleidoscope permite comprender de forma práctica y ordenada los conceptos de lexing, parsing, generación de AST y, especialmente, la generación de código intermedio mediante LLVM, sirviendo como una referencia clara para la construcción de compiladores modernos.

El compilador de Rust pese a su extensión y complejidad está excelentemente documentado, lo cual permite comprender sin demasiado esfuerzo como está estructurado y su funcionamiento interno. Este recurso me ha permitido contemplar como funciona un verdadero compilador del más alto nivel.