\apendice{Documentación técnica de programación}

\section{Introducción}
Este anexo tiene como objetivo servir de guía técnica para desarrolladores que deseen comprender, mantener, extender o colaborar en el proyecto. Se describe la estructura del código fuente, las decisiones de diseño más relevantes, los requisitos técnicos, el proceso de compilación y ejecución, así como el sistema de pruebas existente.

La documentación está orientada a programadores con conocimientos previos de C++ y compiladores, y pretende facilitar la continuidad del proyecto sin necesidad de conocer en profundidad su desarrollo original.

\section{Manual del programador}
El objetivo de este manual es ofrecer una visión técnica del código fuente, de manera que cualquier desarrollador que necesite mantener, extender o comprender el sistema pueda hacerlo de forma progresiva y estructurada.

\subsection{Requisitos previos}
Para poder compilar y trabajar con el proyecto es necesario disponer de los siguientes elementos:

\begin{itemize}
    \item Compilador C++ compatible con el estándar C++17.
    \item LLVM instalado en el sistema (Versión 18).
    \item ANTLR4:
    \begin{itemize}
        \item \eng{Java Runtime Environment} (JRE) para la herramienta ANTLR.
        \item \eng{Runtime} de ANTLR4 para C++ instalado en el sistema.
    \end{itemize}
    \item CMake (versión mínima 3.10).
    \item GoogleTest para la ejecución de las pruebas unitarias.
\end{itemize}

\subsection{Lenguaje de programación y estilo}
El compilador está desarrollado íntegramente en \textbf{C++17}, empleando un estilo de programación modular y orientado a objetos. Este enfoque permite una clara separación de responsabilidades entre los distintos componentes del sistema (análisis léxico, sintáctico, semántico y generación de código).

Como se ha explicado en el capítulo 4 de la memoria, se ha utilizado el estándar de estilos propuesto por LLVM, siendo algunas de sus más apreciables reglas: 

Se siguen las siguientes convenciones de estilo:
\begin{itemize}
    \item \textit{PascalCase} para nombres de clases.
    \item \textit{camelCase} para métodos y variables.
    \item Uso explícito de \texttt{const} siempre que es posible.
    \item Preferencia por \texttt{std::unique\_ptr} y \texttt{std::shared\_ptr} para la gestión segura de memoria.
\end{itemize}

El código está documentado mediante comentarios en formato \textbf{Doxygen}, lo que permite generar documentación automática a partir del código fuente.

\subsection{Convenciones internas y organización del código}
El proyecto sigue una estructura basada en capas, donde cada módulo corresponde a una fase concreta del compilador:

\begin{itemize}
  \item \textbf{\texttt{src/grammar/}}:  
  Contiene la gramática del lenguaje definida mediante ANTLR4, así como el lexer y parser generados.

  \item \textbf{\texttt{src/AST/}}:  
  Implementa la estructura del Árbol de Sintaxis Abstracta (AST) y el visitante encargado de construirlo a partir del árbol sintáctico.

  \item \textbf{\texttt{src/semantic/}}:  
  Contiene el análisis semántico, incluyendo la tabla de símbolos, la gestión de ámbitos (\textit{scopes}) y la detección de errores semánticos.

  \item \textbf{\texttt{src/LLVM/}}:  
  Implementa el generador de código intermedio LLVM IR mediante el patrón \textit{visitor}, así como la integración con el backend de LLVM.

  \item \textbf{\texttt{src/compiler/}}:  
  Orquesta el flujo completo del compilador, desde la lectura del archivo fuente hasta la generación del ejecutable final.

  \item \textbf{\texttt{src/runtime/}}:  
  Contiene el sistema de ejecución en tiempo de ejecución (\eng{runtime}), incluyendo la gestión de eventos, temporización y funciones auxiliares.

  \item \textbf{\texttt{tests/}}:  
  Incluye los tests unitarios implementados con GoogleTest.

  \item \textbf{\texttt{tests/input}}:  
  Incluye ejemplos de diferentes códigos fuente. Estos son utilizados por los tests unitarios.

  \item \textbf{\texttt{doc/}}:  
  Contiene la documentación del proyecto, incluida la memoria del TFG y sus anexos.
\end{itemize}

\imagen{diagramaDirectorios}{Diagrama de directorios del proyecto}

Esta organización permite que un desarrollador pueda trabajar en una fase concreta del compilador sin necesidad de modificar el resto del sistema.

\section{Compilación, instalación y ejecución del proyecto}

\subsection{Compilación}
El proyecto utiliza \textbf{CMake} como sistema de construcción. El fichero \texttt{CMakeLists.txt} define todas las dependencias, objetivos y opciones de compilación necesarias.

El proceso de compilación estándar es el siguiente:

\begin{verbatim}
mkdir build
cd build
cmake ..
make
\end{verbatim}

Este proceso genera el ejecutable del compilador, así como los objetos necesarios del \eng{runtime}.

\subsection{Instalación}
El proyecto no requiere un proceso de instalación formal. Una vez compilado mediante el \eng{script} \code{./build.sh} o el proceso de \eng{make}, el ejecutable puede copiarse a cualquier directorio del sistema.

De forma opcional, el desarrollador puede añadir la ruta del ejecutable a las variables de entorno o definir un alias para facilitar su uso durante el desarrollo.

\subsection{Ejecución}
El compilador se ejecuta desde línea de comandos indicando como argumento el archivo fuente a compilar. Opcionalmente, se pueden añadir distintas banderas para depuración, visualización del AST o generación de LLVM IR. Estas vienen indicadas en la ayuda del programa de la siguiente forma:

\begin{verbatim}
  -h, --help      shows help message and exits
  -v, --version   prints version information and exits
  -o, --output    Output object file [nargs=0..1] [default: "out"]
  --visualizeAST  Generates a AST visualization -pdf file
  --debug         Shows debug data about all the compiler phases.
  --basic         Skips the optimization phase over the LLVM IR module.
  -IR             Generates a LLVM IR file. [nargs=0..1] [default: ""]
\end{verbatim}

Durante el desarrollo, se recomienda ejecutar el compilador en modo \texttt{-{}-debug} para facilitar el análisis del flujo interno del sistema.

\section{Pruebas del sistema}
Para garantizar la estabilidad del sistema ante modificaciones o extensiones, el proyecto incluye un conjunto de pruebas unitarias implementadas mediante \textbf{GoogleTest}.

Las pruebas pueden ejecutarse tras la compilación mediante los objetivos generados por CMake, permitiendo validar que los cambios introducidos no rompen el comportamiento previo del compilador.
