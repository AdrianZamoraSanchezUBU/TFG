\apendice{Documentación de usuario}
\section{Introducción}
Este anexo tiene el propósito de guiar a un usuario final que desee instalar, configurar y usar el compilador. Se describen los requisitos que debe cumplir el usuario, el proceso de Instalación y las instrucciones fundamentales para poder utilizar el compilador de forma correcta.

\section{Requisitos de usuarios}
Antes de comenzar a utilizar el sistema, el usuario debe contar con los siguientes requisitos:

\subsection{Conocimientos}
\begin{itemize}
    \item Conocimientos básicos de programación
    \item Familiaridad con un editor de código
    \item Conocimiento de uso de programas CLI
\end{itemize}

\subsection{Requisitos de hardware} 
\begin{itemize}
    \item CPU: 1GHz
    \item RAM: 1GB
    \item Espacio en disco: 500MB
\end{itemize}

\subsection{Requisitos de software} 
\begin{itemize}
    \item Sistema operativo Windows, Linux o macOS.
    \item Entorno C++
\end{itemize}

\section{Instalación}
La descarga se realizará desde el repositorio de código en este enlace. 

Una completada la descargada se puede colocar el compilador en la ruta que se desee y utilizar esa misma ruta para ejecutar el compilador, o bien introducir la ruta al compilador como variable de entorno en sistemas Windows o como un alias en sistemas Linux, de forma que podamos ejecutar más comodamente el compilador.

\subsection{Variables de entorno (Windows)}

\subsection{Alias (Linux)}

\section{Manual del usuario}
Para compilar, requerimos de tener un fichero con código y podremos ejecutar el compilador con las siguietnes opciones:
\begin{itemize}
    \item \textbf{Compilación básica:}
    \item \textbf{Comprobación de validez léxica:}
    \item \textbf{Comprobación de validez sintática:}
    \item \textbf{Compilación a IR:} 
\end{itemize}