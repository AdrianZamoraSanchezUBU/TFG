\apendice{Documentación de usuario}

\section{Introducción}
Este anexo tiene el propósito de guiar a un usuario final que desee instalar, configurar y usar el compilador. Se describen los requisitos que debe cumplir el usuario, el proceso de instalación y las instrucciones fundamentales para poder utilizar el compilador de forma correcta.

\section{Requisitos de usuarios}
Antes de comenzar a utilizar el sistema, el usuario debe contar con los siguientes requisitos:

\subsection{Conocimientos}
\begin{itemize}
    \item Conocimientos básicos de programación.
    \item Familiaridad con algún editor de código o entorno de desarrollo.
    \item Conocimiento básico del uso de programas mediante línea de comandos (CLI).
\end{itemize}

\subsection{Requisitos de hardware}
\begin{itemize}
    \item CPU: Procesador de al menos 1 GHz.
    \item RAM: 1 GB de memoria RAM.
    \item Espacio en disco: 500 MB libres.
\end{itemize}

\subsection{Requisitos de software}
\begin{itemize}
    \item Sistema operativo Windows, Linux o macOS.
    \item Compilador C++ compatible con el estándar C++17.
    \item LLVM instalado en el sistema.
\end{itemize}

\section{Instalación}
La descarga del compilador se realiza desde el repositorio de código del proyecto. Una vez descargado, el usuario puede compilar el proyecto siguiendo las instrucciones proporcionadas en el repositorio.

Tras completar la compilación, el ejecutable del compilador puede colocarse en cualquier directorio del sistema. Para facilitar su uso, se recomienda añadir la ruta del ejecutable a las variables de entorno del sistema o definir un alias, según el sistema operativo utilizado.

\subsection{Variables de entorno (Windows)}
En sistemas Windows, se puede añadir la carpeta que contiene el ejecutable del compilador a la variable de entorno \texttt{PATH} siguiendo estos pasos:

\begin{itemize}
    \item Abrir el panel de \emph{Configuración del sistema}.
    \item Acceder a \emph{Variables de entorno}.
    \item Editar la variable \texttt{PATH} del sistema o del usuario.
    \item Añadir la ruta donde se encuentra el ejecutable del compilador.
\end{itemize}

Una vez realizado este proceso, el compilador podrá ejecutarse desde cualquier terminal de comandos sin necesidad de indicar su ruta completa.

\subsection{Alias (Linux)}
En sistemas Linux o macOS, se puede crear un alias para facilitar la ejecución del compilador. Para ello, basta con añadir la siguiente línea al archivo \texttt{\textasciitilde/.bashrc} o \texttt{\textasciitilde/.zshrc}:

\begin{verbatim}
alias Tcompiler="/ruta/al/ejecutable/TCompiler"
\end{verbatim}

Tras recargar la configuración del terminal, el compilador podrá ejecutarse escribiendo simplemente \texttt{Tcompiler}.

\section{Manual del usuario}
Para compilar un programa escrito en el lenguaje, es necesario disponer de un fichero fuente con el código a compilar. El compilador se ejecuta desde la línea de comandos utilizando la siguiente sintaxis general:

\begin{verbatim}
Tcompiler <archivo_fuente> [opciones]
\end{verbatim}

A continuación se describen las principales opciones disponibles:

\begin{itemize}
    \item \texttt{Tcompiler -h / -{}-help:}  
    Muestra un mensaje de ayuda con la descripción de las opciones disponibles.

    \item \texttt{Tcompiler <archivo fuente>:}  
    Compila el archivo fuente especificado y genera un archivo objeto como salida.

    \item \texttt{Tcompiler <archivo fuente> -{}-debug:}  
    Activa el modo de depuración, mostrando información adicional como la lista de tokens, la tabla de símbolos y el código LLVM IR generado.

    \item \texttt{Tcompiler <archivo fuente> -{}-visualizeAST:}  
    Genera una representación visual del Árbol de Sintaxis Abstracta (AST) en formato PDF.

    \item \texttt{Tcompiler <archivo fuente> -IR <archivo>:}  
    Exporta el código LLVM IR generado al archivo especificado.

    \item \texttt{Tcompiler <archivo fuente> -o <programa a generar>:}  
    Permite especificar el nombre del archivo objeto o ejecutable generado por el compilador.
\end{itemize}
