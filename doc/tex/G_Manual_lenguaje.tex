\apendice{Manual del lenguaje T}

\section{Introducción}

El lenguaje T es un lenguaje de programación imperativo y de tipado estático, con una sintaxis inspirada en lenguajes de la familia C. Su principal rasgo distintivo es la incorporación de \emph{eventos temporales} y \emph{eventos condicionados}, que permiten programar comportamientos reactivos basados en el tiempo o en condiciones lógicas del programa.

Un programa en T se compone de declaraciones de variables, expresiones, estructuras de control, definiciones de funciones y definiciones de eventos, los cuales pueden ejecutarse de forma automática según las reglas temporales o condicionales especificadas.

Este capítulo describe el lenguaje desde el punto de vista del programador, detallando su sintaxis y semántica básica, sin entrar en aspectos internos del compilador.

\section{Estructura de un programa}

Un programa en T consiste en una secuencia de sentencias que se ejecutan en el orden en el que aparecen. No existe una función \texttt{main} explícita; el bloque principal del programa actúa como punto de entrada.

\begin{minted}{c}
int a;
a = 10;

print(a);
return 0;
\end{minted}

La instrucción \texttt{return} puede aparecer en el bloque principal, indicando el valor de salida del programa.

\section{Comentarios y formato}

\subsection{Comentarios}

T admite comentarios de una sola línea, iniciados por \texttt{//}. Todo el texto desde este símbolo hasta el final de la línea es ignorado por el compilador.

\begin{minted}{c}
// Esto es un comentario
int a;
\end{minted}

\subsection{Espacios en blanco}

Los espacios, tabuladores y saltos de línea no tienen significado semántico y se utilizan únicamente para mejorar la legibilidad del código.

\subsection{Identificadores}

Los identificadores representan nombres de variables, funciones y eventos. Deben comenzar por una letra y pueden contener letras y dígitos.

Ejemplos válidos:
\begin{itemize}
    \item \texttt{a}
    \item \texttt{contador1}
    \item \texttt{miFuncion2}
\end{itemize}

\section{Tipos de datos}

El lenguaje T define los siguientes tipos primitivos:

\begin{itemize}
    \item \texttt{int}
    \item \texttt{float}
    \item \texttt{char}
    \item \texttt{string}
    \item \texttt{bool}
    \item \texttt{void}
    \item \texttt{time}
\end{itemize}

Además, existe el modificador \texttt{ref}, utilizado exclusivamente para indicar paso de parámetros por referencia en funciones.

\section{Literales}

\subsection{Literales numéricos}

Los literales enteros y reales se escriben de la forma habitual:

\begin{minted}{c}
10
3.14
\end{minted}

\subsection{Literales de cadena}

Las cadenas de texto se escriben entre comillas dobles:

\begin{minted}{c}
"Ejemplo de string"
\end{minted}

\subsection{Literales booleanos}

El lenguaje define dos literales booleanos:

\begin{minted}{c}
true
false
\end{minted}

\subsection{Literales temporales}

El tipo \texttt{time} representa valores temporales expresados mediante una cantidad numérica seguida de una unidad. Las unidades disponibles son:

\begin{itemize}
    \item \texttt{tick}
    \item \texttt{sec}
    \item \texttt{min}
    \item \texttt{hr}
\end{itemize}

Ejemplos:

\begin{minted}{c}
5 sec
2.5 min
10 tick
1 hr
\end{minted}

Durante el proceso de compilación todos los valores son transformados a \emph{tick}, cada tick equivale a 100 ms

\section{Variables}

\subsection{Declaración de variables}

Las variables se declaran especificando su tipo y su identificador:

\begin{minted}{c}
int a;
char a;
float x;
string name;
bool ok;
time t;
\end{minted}

\subsection{Asignación}

Una variable puede recibir un valor mediante el operador de asignación \texttt{=}:

\begin{minted}{c}
a = 10;
x = 2.718;
name = "Adrian";
ok = true;
\end{minted}

También es posible declarar e inicializar una variable en una única sentencia:

\begin{minted}{c}
int a = 10;
float x = 3.14;
\end{minted}

\section{Expresiones}

\subsection{Expresiones aritméticas}

T soporta los operadores aritméticos básicos: suma \texttt{+}, resta \texttt{-}, multiplicación \texttt{*}, división \texttt{/} y módulo \texttt{\%}.

\begin{minted}{c}
int a = 2 + 3 * 4;
float b = (1.0 + 2.0) / 3.0;
\end{minted}

Los paréntesis permiten modificar la precedencia de evaluación.

\subsection{Expresiones relacionales}

Se admiten las comparaciones:

\begin{center}
\texttt{== \quad != \quad < \quad <= \quad > \quad >=}
\end{center}

\begin{minted}{c}
if (a < 10) {
    print(a);
}
\end{minted}

\section{Operadores de incremento y decremento}

El lenguaje admite operadores de incremento y decremento tanto en forma prefija como postfija:

\begin{minted}{c}
a++;
--a;
\end{minted}

Estos operadores pueden formar parte de expresiones más complejas.

\section{Estructuras de control}

\subsection{Condicionales}

La estructura condicional se define mediante \texttt{if} y \texttt{else}:

\begin{minted}{c}
if (a < 0) {
    print("Negativo");
} else if (a < 10) {
    print("Pequeño");
} else {
    print("Grande");
}
\end{minted}

\subsection{Bucles}

\subsubsection{Bucle \texttt{while}}

\begin{minted}{c}
while (a < 10) {
    a++;
}
\end{minted}

\subsubsection{Bucle \texttt{for}}

El bucle \texttt{for} tiene la forma:

\begin{minted}{c}
for (int i = 0; i < 10; i = i + 1) {
    print(i);
}
\end{minted}

\subsection{Control de bucle}

Se permiten las sentencias de control en bucles:

\begin{itemize}
    \item \texttt{break}: Termina el bucle
    \item \texttt{continue}: Vuelve a evaluar la condición
\end{itemize}

\section{Funciones}

\subsection{Declaración y definición}

Una función puede declararse sin cuerpo:

\begin{minted}{c}
int function suma(int a, int b);
\end{minted}

O definirse completamente:

\begin{minted}{c}
int function suma(int a, int b) {
    return a + b;
}
\end{minted}

\subsection{Parámetros por referencia}

El modificador \texttt{ref} permite pasar parámetros por referencia:

\begin{minted}{c}
void function inc(ref int a) {
    a = a + 1;
}
\end{minted}

\section{Eventos}

Los eventos constituyen la característica diferenciadora del lenguaje T. Un evento define un bloque de código que se ejecuta automáticamente según una condición temporal o lógica.

\subsection{Eventos temporales}

Los eventos temporales se definen mediante los comandos \texttt{every}, \texttt{after} y \texttt{at}.

\begin{minted}{c}
event tickLogger every 1 sec {
    print("tick");
}
\end{minted}

\begin{minted}{c}
event bomba after 5 sec {
    print("boom");
}
\end{minted}

En el estado actual del proyecto, \texttt{every} se encuentra plenamente implementado y soportado por el compilador y el \emph{runtime}. Por otro lado, las construcciones \texttt{after} y \texttt{at} han sido definidas y reconocidas por el lenguaje, pero su implementación completa no ha sido abordada debido a limitaciones de tiempo. No obstante, se han reservado como parte de la sintaxis del lenguaje con el objetivo de facilitar su incorporación en futuras ampliaciones.

\subsection{Límite de ejecuciones}

Un evento puede limitar su número de ejecuciones mediante \texttt{limit}:

\begin{minted}{c}
event hello every 1 sec limit 3 {
    print("hola");
}
\end{minted}

\subsection{Finalización de eventos}

Dentro de un evento puede utilizarse la instrucción \texttt{exit} seguido del identificador de un evento, finalizando asi este evento en concreto:

\begin{minted}{c}
exit tickLogger;
\end{minted}

\section{Limitaciones actuales del lenguaje}

En su estado actual, el lenguaje presenta las siguientes limitaciones:

\begin{itemize}
    \item No se soportan estructuras de datos complejas como arrays o \texttt{struct}.
    \item El tipo \texttt{ref} solo está disponible en parámetros de funciones. Puesto que el paso en eventos requiere de gestión avanzada de memoria.
    \item No existe mecanismos para la definición o inclusión de bibliotecas. 
    \item Por cuestión de tiempo, los eventos no han podido enriquecerse con otros mecanismos de ejecución.
\end{itemize}
