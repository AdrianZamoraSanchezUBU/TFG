\apendice{Anexo de sostenibilización curricular}

\section{Introducción}
En este anexo se abordan los conceptos de sostenibilidad, es decir, la busqueda de un equilibrio entre el desarrollo económico, la protección del medio ambiente y la justicia social, con el objetivo de garantizar un futuro digno para las próximas generaciones. 

En el marco de este Trabajo de Fin de Grado, centrado en el desarrollo de un lenguaje de programación y la construcción de su compilador, resulta complejo reflexionar sobre como herramientas como esta, la cual aparentemente queda fuera de estos debates sociales y ambientales, puede tambien contribuir a los objetivos de sostenibilidad, sin embargo, hay algunos puntos en los que podemos abordar este tema.

\section{Principios de sostenibilidad del proyecto}
A continuación se tratan los principios de sostenibilidad desde el punto de vista de este proyecto:
Principio ético: la programación de un compilador debe orientarse a respetar la transparencia, la seguridad de los datos y la accesibilidad del software. La forma más accesible de presentar este software es con una licencia que lo haga gratuito e impulse a los usuarios a utilizarlo en todo tipo de proyectos.

Principio holístico: el compilador no debe entenderse como un simple traductor de lenguaje humano a lenguaje máquina, sino como una pieza de software que tiene un posible impacto en la eficiencia energética de un sistema informático y que puede afectar a la productividad de los desarrolladores.

Principio de complejidad: un software moderno como este debe ser pensado en dimensiones sociales y ambientales, asegurando que el compilador sea eficiente reduciendo el consumo eléctrico así como evitar que el usuario necesito contar con un hardware de altas prestaciones para su ejecución, lo que a largo plazo puede tener un impacto ambiental.

Principio de glocalización: en el marco de la creación de software, podemos incluir este principio tratando de hacer llegar este software a los usuario en sus lenguas locales y con documentación adaptada a todos los niveles de conocimiento técnico posibles, asegurando sus accesibilidad.

Principio de transversalidad: el criterio de la sostenibilidad no es un factor aislado a algunos de los objetivos del proyecto, sino que debe estar presente durante todo ciclo de diseño así como en el proceso de desarrollo, asegurando todos los principios cubiertos anteriormente.

\section{Competencias transversales en sostenibilidad}
Siguiendo lo indicado en las directrices de la CRUE, se pueden identificar las siguientes competencias transversales vinculadas al desarrollo del compilador:

SOS1: Contextualización crítica del conocimiento. El compilador se ha diseñado no solo como un ejercicio académico, sino también como una herramienta que puede integrarse en contextos reales de desarrollo de software. Esta visión permite relacionar el trabajo con problemáticas más amplias, como la eficiencia energética de los centros de datos o la necesidad de software más accesible.

SOS2: Uso sostenible de recursos. Un compilador eficiente contribuye a que los programas generados consuman menos recursos. Esto tiene un impacto directo en el consumo energético de los dispositivos, lo cual es especialmente relevante en un contexto donde los centros de procesamiento de datos representan una parte significativa del gasto energético global.

SOS3: Participación comunitaria. El software desarrollado puede ponerse a disposición de la comunidad universitaria o profesional, fomentando la colaboración abierta y el aprendizaje compartido mediante licencias como la MIT, lo que promueve el acceso igualitario al conocimiento.

SOS4: Principios éticos. En la construcción del compilador se ha procurado mantener buenas prácticas de programación, así como transparencia en el funcionamiento de la herramienta, evitando prácticas que puedan considerarse opacas o poco éticas.

\section{Impacto social y ambiental del proyecto}

El impacto de un compilador sobre la sostenibilidad puede analizarse en dos niveles:

Impacto ambiental: la optimización del código y la reducción del consumo de recursos computacionales contribuyen indirectamente a disminuir el gasto energético. Aunque este efecto pueda parecer limitado en un solo proyecto, los programas generador por los compiladores pueden influir directamente en el consumo de energía de millones de dispositivos.

Impacto social: un compilador bien diseñado puede facilitar el acceso a la programación a estudiantes y profesionales de distintas procedencias. La posibilidad de que la herramienta se difunda como recurso abierto fomenta la democratización del conocimiento, reduciendo barreras de entrada en el aprendizaje de lenguajes de programación.

\section{Conclusión}
Aunque a primera vista este proyecto pueda parecer desconectado de las cuestiones de sostenibilidad, un análisis más detallado demuestra que es posible establecer relaciones significativas. La eficiencia computacional, la ética en el diseño de software, la accesibilidad y la potencial contribución al aprendizaje colectivo son aspectos que reflejan cómo la informática puede integrarse en un marco más amplio de desarrollo humano sostenible.

De esta manera, el presente trabajo no solo cumple una función académica y técnica, sino que también se alinea con los compromisos institucionales de la universidad en materia de sostenibilidad, reforzando la idea de que toda actividad de investigación y desarrollo debe contemplar su impacto social, económico y ambiental.