\capitulo{2}{Objetivos del proyecto}

\begin{itemize}
    \item \textbf{Diseñar un lenguaje de programación}: 
    Definir la sintaxis y semántica de un nuevo lenguaje, estableciendo sus estructuras básicas, tipos de datos, operadores y reglas gramaticales.

    \item \textbf{Desarrollar un compilador capaz de reconocer el lenguaje}: 
    Implementar un compilador que analice y procese el código fuente del lenguaje, abarcando las fases de análisis léxico, sintáctico y semántico.

    \item \textbf{Estructurar de forma modular el proceso de compilación}: 
    Separar las diferentes fases de compilación manteniendo las responsabilidades bien definidas dentro de cada una de las fases, evitando así una sobrecarga de responsabilidades que dificulte realizar ampliaciones o modificaciones.

    \item \textbf{Generación y gestión de errores}: 
    Incorporar un sistema de detección y reporte de errores léxicos, sintácticos y semánticos, proporcionando mensajes claros y detallados que faciliten la depuración del código.

    \item \textbf{Herramienta de análisis y depuración visual}: 
    Desarrollar una utilidad visual que permita representar y explorar estructuras internas del compilador, como el Árbol de Sintaxis Abstracta (AST), facilitando la comprensión y depuración del lenguaje.

    \item \textbf{Generación de una representación intermedia (IR)}: 
    Implementar la traducción del lenguaje diseñado a una representación intermedia basada en LLVM IR, permitiendo la optimización y posterior generación de código máquina para distintas arquitecturas.

    \item \textbf{Implementación de utilidades internas del lenguaje}: 
    Desarrollar un conjunto de funciones o bibliotecas básicas que proporcionen utilidades estándar dentro del entorno del lenguaje diseñado.
\end{itemize}

\subsection{Objetivos técnicos}

El desarrollo del proyecto se apoya en el uso de herramientas y tecnologías ampliamente reconocidas en el ámbito de los compiladores modernos:

\begin{itemize}
    \item \textbf{ANTLR}: Se utiliza ANTLR como generador de analizadores léxicos y sintácticos, dada su modernidad, robustez y amplia adopción en el desarrollo de lenguajes formales.

    \item \textbf{LLVM}: Se emplea LLVM como infraestructura de compilación, debido a su capacidad para generar una representación intermedia (IR) portable y optimizable, independiente de la arquitectura de hardware.

    \item \textbf{C++}: El proyecto se desarrolla con el lenguaje C++, por su excelente integración tanto con ANTLR (a través del ANTLR C++ Runtime) como con LLVM, además de sus ventajas en programación orientada a objetos y gestión eficiente de memoria.

    \item \textbf{LaTex forest}: Esta herramienta ampliamente utilizada en el ámbio académico se utiliza para visualizaciones y diagramas. 
\end{itemize}