\capitulo{2}{Objetivos del proyecto}

\begin{itemize}
    \item \textbf{Diseñar un lenguaje de programación}: 
    Definir nuevo lenguaje de programación, estableciendo sus estructuras básicas, tipos de datos, operadores y reglas gramaticales.

    \item \textbf{Desarrollar un compilador capaz de reconocer el lenguaje}: 
    Implementar un compilador con un diseño modular y moderno, capaz de analizar y procesar el código fuente del lenguaje, abarcando las fases de análisis léxico, sintáctico, semántico, generación de una representación intermedia (IR por sus siglas en inglés), optimización y finalmente generación de un ejecutable.

    \item \textbf{Estructurar de forma modular el proceso de compilación}: 
    Separar las diferentes fases de compilación manteniendo las responsabilidades bien definidas dentro de cada una de las fases, evitando así una sobrecarga de responsabilidades que dificulte realizar ampliaciones o modificaciones.

    \item \textbf{Generación y gestión de errores}: 
    Incorporar un sistema de detección y reporte de errores léxicos, sintácticos y semánticos, proporcionando mensajes claros y detallados que faciliten la depuración del código. 

    \item \textbf{Herramienta de análisis y depuración visual}: 
    Desarrollar una utilidad visual que permita representar y explorar estructuras internas del compilador, como el árbol de sintaxis abstracta o la tabla de símbolos, facilitando la comprensión y depuración del lenguaje.

    \item \textbf{Generación de IR basada en el ecosistema LLVM}: 
    Implementar la traducción del lenguaje diseñado a una representación intermedia basada en LLVM IR \cite{lattner2004llvm}, permitiendo la optimización y posterior generación de código máquina para distintas arquitecturas.

    \item \textbf{Implementación de una biblioteca estándar}: 
    Desarrollar una pequeña librería estándar, es decir, un conjunto de funciones básicas que proporcionen utilidades estándar dentro del lenguaje diseñado.

    \item \textbf{Proporcionar al menos una funcionalidad específica del lenguaje}: 
    Se trabajará en la implementación de un sistema de ejecución de código basado en eventos y tiempo. Este será un factor diferenciador y experimental frente a otros lenguajes de propósito general.
    
    \item \textbf{Proporcionar un runtime personalizado}:
    Para poder implementar las funcionalidades específicas del lenguaje e indagar en otro aspecto académico de los lenguajes de programación, se implementará un runtime diseñado específicamente para poder ejecutar el código generado.  
\end{itemize}
