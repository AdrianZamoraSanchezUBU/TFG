\capitulo{7}{Conclusiones y Líneas de trabajo futuras}
\subsection{Conclusiones}
Tras realizar el proyecto, he conseguido una comprensión profunda sobre como funcionan los lenguajes de programación así como los procesos de compilación. He lanzado una mirada crítica a herramientas y lenguajes que llevo usando desde hace muchos años pero nunca me había preguntado hasta tal punto como operaban fuera de la mirada del programador. 

Tras comprender el trabajo tan complejo que supone crear un compilador de la calidad de GCC o interpretes como javac, me siento muy agradecido de tener acceso a tecnologías tan punteras y sofisticadas que nos permiten desarrollar de forma tan cómoda nuestra labor en el desarrollo de software.

\subsection{Lineas de trabajo futuras}
Puesto que el proyecto está extremadamente acotado por su complejidad y el tiempo disponible, considero que hay numerosas lineas de trabajo con las que me gustaría continuar próximamente, algunas de las mas destacables:

- Migrar de ANTLR a un lexer y parsers propios. Pese a que ANTLR4 tiene una calidad notable, la mayoría de grandes compiladores actuales proporcionan sus propios lexers y parsers, puesto que son mucho más optimizables y personalizables. 
- Ampliar el lenguaje con POO, una de las metodologías de la programación más populares y de las que mejor encajan con un lenguaje que pretende modelar eventos.
- Un runtime más sofisticado, con mayor precisión en el control del tiempo y rutinas más optimizadas hechas directamente en ensamblador.
- Mas según lo que de tiempo o no a implementar...