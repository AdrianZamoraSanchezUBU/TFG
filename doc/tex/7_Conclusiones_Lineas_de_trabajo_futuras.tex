\capitulo{7}{Conclusiones y líneas de trabajo futuras}
\subsection{Conclusiones}
Tras realizar el proyecto, he conseguido una comprensión profunda sobre como funcionan los lenguajes de programación así como los procesos de compilación y en menor medida el proceso de interpretación. Este trabajo me ha hecho lanzar una mirada crítica a herramientas y lenguajes que llevo usando desde hace muchos años pero nunca me había preguntado realmente como operaban fuera de la mirada del programador. 

Estudiar, analizar y comprender el trabajo tan extenso y complejo que supone crear un compilador de la calidad de GCC o rustc, también me ha hecho que me sienta especialmente agradecido de disponer de acceso a tecnologías tan punteras y sofisticadas de forma totalmente gratuita, estas herramientas nos permiten desarrollar de forma cómoda nuestra labor en el desarrollo de software sin pedir nada a cambio.

\subsection{Líneas de trabajo futuras}
Puesto que el proyecto está extremadamente acotado por su complejidad y el tiempo disponible, considero que hay numerosas lineas de trabajo con las que me gustaría continuar próximamente, algunas de las mas destacables:

- Migrar de ANTLR a un lexer y parsers propios. Pese a que ANTLR4 genera analizadores de una calidad comparable a cualquier compilador profesional, la mayoría de grandes compiladores actuales proporcionan sus propios lexers y parsers, puesto que son mucho más optimizables y personalizables. Además, evitan la inclusión de clases, contextos y símbolos innecesarios que aumentan en algunos casos innecesariamente la complejidad de los analizadores.
- Ampliar el lenguaje para soportar la metodología OOP, una de las metodologías de la programación más populares y de las que mejor encajan con un lenguaje que pretende modelar estructuras tan complejas como eventos.
- Un runtime más sofisticado, con mayor precisión en el control del tiempo y rutinas más optimizadas hechas directamente en ensamblador.
- Mas según lo que de tiempo o no a implementar...