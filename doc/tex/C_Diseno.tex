\apendice{Especificación de diseño} %TODO

\section{Introducción}
En esta sección se describe cómo se construirá el sistema desde el punto de vista del diseño. El objetivo es definir la estructura general de la aplicación, los modelos de datos que la sustentan y los procedimientos que regirán su funcionamiento. De esta manera se establece un puente entre los requisitos previamente definidos y la posterior implementación, garantizando que el desarrollo siga una organización clara y coherente.

\section{Diseño de datos} 
\subsection{Estructuras de datos utilizadas}
AST

Tabla de símbolos

\subsection{Diagramas UML de los objetos del sistema}

\subsection{Tipos de datos del lenguaje}


\section{Diseño arquitectónico} % TODO
A continuación se lista los módulos principales:

Analizador léxico o lexer

Analizador sintáctico o parser

Analizador semántico

Generador de código IR

El siguiente diagrama muestra linealmente una ejecución sin errores, donde se puede visualizar el estado de los datos en cada capa:
\imagen{diagramaFasesCompilador}{Diagrama fases del compilador}

\section{Diseño procedimental} % TODO
A continuación se explican de qué formas realizan sus tareas los elementos del diseño arquitectónico visto anteriormente:

Algoritmo del lexer (NFA → DFA)

Algoritmo parser (Adaptive LL()* / ALL(*)) que primero intenta decisiones con bajo lookahead (SLL), y si falla, usa contexto completo (modo LL)

Algoritmo para generar la AST (visitor pattern sobre contextos ANTLR) 

También visitor pattern para recorrer AST en análisis semántico y generación de IR.

Algoritmo de scheduling de eventos
